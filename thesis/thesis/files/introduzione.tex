\chapter{Introduzione}\label{ch:introduzione}
\ldots
%\cite{gruntzig1978transluminal}

% Contesto
\section{Contesto}\label{sec:contesto}
Il monitoraggio della qualità dell'aria è una delle attività più importanti per la tutela della salute pubblica. La qualità dell'aria può essere influenzata da molte sorgenti di emissione, tra cui le automobili, le centrali elettriche, gli impianti di riscaldamento e le fabbriche. I principali inquinanti atmosferici sono il biossido di zolfo, gli idrocarburi policiclici aromatici, il monossido di carbonio e gli ozono. Gli effetti dell'inquinamento atmosferico sulla salute sono molteplici e possono essere a breve o a lungo termine. I principali rischi sono l'asma, le malattie cardiovascolari, il cancro e le malattie respiratorie. Il monitoraggio della qualità dell'aria permette di individuare le sorgenti di emissione e di intervenire per ridurre l'inquinamento atmosferico.

\subsection{Descrizione del problema}\label{ssec:problema}
\ldots

\subsection{Progetti simili}\label{ssec:competitor}

\begin{itemize}
	\item \textbf{Airly} (https://airly.org/) è una piattaforma che consente di condividere informazioni ambientali in tempo reale, grazie alla quale è possibile monitorare la qualità dell'aria e i livelli di inquinamento;
\end{itemize}

%Airly, Aqicn, IQAir, Decentlab, SMART Treedom, PlanetWatch, HackAIR

% La piattaforma AirQino
\section{La piattaforma AirQino}\label{sec:airqino}
\ldots

\subsection{Hardware dei sensori}\label{ssec:hardware}
\ldots

\subsection{Architettura e tecnologie}\label{ssec:airqino-architettura}
\ldots

\subsection{Progetti correlati}\label{ssec:correlati}
\ldots