\chapter{Sviluppi tecnologici}\label{ch:sviluppi}

% Replica del database di produzione
\section{Replica del database di produzione}\label{sec:replica}
\ldots

\subsection{Motivazioni}\label{ssec:replica-motivazioni}
\ldots

\subsection{Streaming Replication}\label{ssec:streaming-replication}
La streaming replication di PostgreSQL è una funzionalità che consente di replicare i dati in tempo reale da una istanza di PostgreSQL a un'altra. Questo significa che, se si modificano i dati in una delle istanze, questi saranno immediatamente replicati anche nell'altra istanza. La streaming replication di PostgreSQL offre diversi vantaggi:

- maggiore disponibilità dei dati: se una delle istanze di PostgreSQL viene a mancare, i dati saranno comunque disponibili nell'altra istanza;

- maggiore velocità di replica: i dati vengono replicati in tempo reale, senza dover attendere il completamento delle operazioni di replica;

- riduzione del carico sulle risorse: la replica in tempo reale riduce il carico sulle risorse della infrastruttura di storage.

La replica si basa sulle transazioni WAL (Write Ahead Log) e utilizza il protocollo TCP per garantire una connessione sicura tra i server.
---

TimescaleDB può gestire la replica utilizzando la streaming replication integrata di PostgreSQL (vedi docs ufficiali: \url{https://docs.timescale.com/timescaledb/latest/how-to-guides/replication-and-ha/replication/}).

\subsubsection{Preparazione del database primario}

\begin{itemize}
  \item Creare un utente PostgreSQL con un ruolo adatto ad avviare la streaming replication:
   \begin{lstlisting}[language=sql, caption=TODO]
SET password_encryption = 'scram-sha-256'; 
CREATE ROLE repuser WITH REPLICATION PASSWORD 'SOME_SECURE_PASSWORD' LOGIN;\end{lstlisting}
  \item Aggiungere i seguenti parametri al file \url{/var/lib/postgresql/data/postgresql.conf}:
  \begin{lstlisting}[caption=TODO]
listen_addresses= '*'
wal_level = replica
max_wal_senders = 2
max_replication_slots = 2
synchronous_commit = off
\end{lstlisting}
\end{itemize}

% Ottimizzazione di query temporali
\section{Ottimizzazione di query temporali}\label{sec:cont-aggr}
\ldots

\subsection{Motivazioni}\label{ssec:cont-aggr-motivazioni}
\ldots

\subsection{Continuous Aggregates}\label{ssec:cont-aggr}
I continuous aggregate sono una funzionalità integrata in TimescaleDB che consente di aggregare i dati in tempo reale, senza la necessità di eseguire query aggiuntive. Questa funzionalità utilizza i contatori per tenere traccia dei dati aggregati in tempo reale e fornisce una rappresentazione dei dati aggregati in tempo reale.

I continuous aggregate offrono numerosi vantaggi, tra cui:

- Flessibilità: è possibile aggregare dati in tempo reale in base a qualsiasi criterio desiderato.

- Risparmio di tempo: non è necessario eseguire query aggiuntive per ottenere informazioni aggregate in tempo reale.

- Risparmio di spazio: i dati aggregati in tempo reale occupano meno spazio rispetto ai dati non aggregati.

- Maggiore efficienza: i continuous aggregate sono più efficienti dei query batch per l'aggregazione dei dati in tempo reale.

\subsection{Risultati ottenuti}\label{ssec:cont-aggr-risultati}
\ldots