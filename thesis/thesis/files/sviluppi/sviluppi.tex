\chapter{Sviluppi tecnologici}\label{ch:sviluppi}
Questo capitolo riguarda i miglioramenti realizzati dal punto di vista tecnologico che sono andati direttamente ad impattare la piattaforma AirQino, migliorandone in un caso l'affidabilità dei dati e nell'altro i tempi di risposta del database per query particolarmente onerose.

% Replica del database di produzione
\section{Replica del database di produzione}\label{sec:replica}

Spesso fare analisi mediamente complesse sui dati contenuti in un database può comportare rallentamenti nei tempi di risposta. Se questi carichi risultano frequenti, il sistema può arrivare a bloccarsi e interrompere il servizio.

Una soluzione per risolvere questo problema è la creazione di una (o più) repliche del database primario. Nella replica, i dati e gli oggetti del database vengono copiati e distribuiti su un altro spazio fisico. Le operazioni onerose a questo punto possono essere fatte direttamente sulla replica che agisce come nodo secondario: in questo modo, il carico viene distribuito e non si intaccano le performance del database principale.

Il concetto di \textit{replica} è diverso dal \textit{mirroring}, in cui vengono create una o più copie di un database su diverse istanze del server, e funzionano come copie di riserva (e si attivano soltanto nel caso di guasto del nodo principale).

Un sistema di replica correttamente implementato può offrire diversi vantaggi, tra cui riduzione del carico (perchè i dati replicati possono essere distribuiti su più server), efficienza (i server offrono prestazioni migliori perchè meno gravati da query pesanti) e ridondanza (i dati sono raggiungibili da più indirizzi).

Di contro, questa tecnica comporta la necessità di mantenimento dei nodi secondari, spesso collocati su server diversi (con i costi a questi associati). Inoltre, repliche errate o non implementate in maniera corretta possono causare la mancata sincronizzazione tra i nodi, portando ad una perdita o incoerenza dei dati.

\subsection{Motivazioni}\label{ssec:replica-motivazioni}
La replica offrire vantaggi principalmente legati alle prestazioni, disponibilità e sicurezza dei dati:
\begin{enumerate}
  \item \textbf{Maggiore affidabilità}: tramite la replica del database viene garantita la disponibilità dei dati anche nel caso in cui una delle macchine presenti un guasto hardware. In questo caso, il sistema di gestione del database distribuito deve essere in grado di indirizzare gli utenti interessati ad uno degli altri nodi disponibili;
  \item \textbf{Miglioramento delle prestazioni}: essendo i dati distribuiti su diverse istanze, accessi multipli non saturano i server. Questo aspetto risulta particolarmente importante per applicazioni che possono avere una grande quantità di richieste simultanee;
  \item \textbf{Maggiore sicurezza dei dati}: Mentre in un sistema tradizionale i backup di un database (se effttuati) sono archiviati sullo stesso disco, con la replica del database vengono scritti su più server, aumentandone di fatto l'affidabilità e la ridondanza.
\end{enumerate}

Esistono diverse tecniche di replicazione del database, che dipendono sia dalla tecnologia utilizzata (MySQL, Postgres) che dalla natura del database stesso (relazionale o non relazionale). Il database di AirQino fa uso di Timescale\footnote{Timescale: Time-series data simplified - \url{https://www.timescale.com}}, basato su Postgres; una caratteristica di Postgres è la possibilità di replicazione con la tecnologia di \textbf{Streaming Replication}, descritta di seguito.

\subsection{Streaming Replication}\label{ssec:streaming-replication}
La streaming replication di PostgreSQL è una funzionalità che consente di replicare i dati in tempo reale da una istanza di PostgreSQL a un'altra. Questo significa che, se si modificano i dati in una delle istanze, questi saranno immediatamente replicati anche nell'altra istanza. La streaming replication di PostgreSQL offre diversi vantaggi:

- maggiore disponibilità dei dati: se una delle istanze di PostgreSQL viene a mancare, i dati saranno comunque disponibili nell'altra istanza;

- maggiore velocità di replica: i dati vengono replicati in tempo reale, senza dover attendere il completamento delle operazioni di replica;

- riduzione del carico sulle risorse: la replica in tempo reale riduce il carico sulle risorse della infrastruttura di storage.

La replica si basa sulle transazioni WAL (Write Ahead Log) e utilizza il protocollo TCP per garantire una connessione sicura tra i server.
---

TimescaleDB può gestire la replica utilizzando la streaming replication integrata di PostgreSQL (vedi docs ufficiali: \url{https://docs.timescale.com/timescaledb/latest/how-to-guides/replication-and-ha/replication/}).

\subsubsection{Preparazione del database primario}

\begin{itemize}
  \item Creare un utente PostgreSQL con un ruolo adatto ad avviare la streaming replication:
   \begin{lstlisting}[language=sql, caption=TODO]
SET password_encryption = 'scram-sha-256'; 
CREATE ROLE repuser WITH REPLICATION PASSWORD 'SOME_SECURE_PASSWORD' LOGIN;\end{lstlisting}
  \item Aggiungere i seguenti parametri al file \url{/var/lib/postgresql/data/postgresql.conf}:
  \begin{lstlisting}[caption=TODO]
listen_addresses= '*'
wal_level = replica
max_wal_senders = 2
max_replication_slots = 2
synchronous_commit = off
\end{lstlisting}
\end{itemize}

% Ottimizzazione di query temporali
\section{Ottimizzazione di query temporali}\label{sec:cont-aggr}
\ldots

\subsection{Motivazioni}\label{ssec:cont-aggr-motivazioni}
\ldots

\subsection{Continuous Aggregates}\label{ssec:cont-aggr}
I continuous aggregate sono una funzionalità integrata in TimescaleDB che consente di aggregare i dati in tempo reale, senza la necessità di eseguire query aggiuntive. Questa funzionalità utilizza i contatori per tenere traccia dei dati aggregati in tempo reale e fornisce una rappresentazione dei dati aggregati in tempo reale.

I continuous aggregate offrono numerosi vantaggi, tra cui:

- Flessibilità: è possibile aggregare dati in tempo reale in base a qualsiasi criterio desiderato.

- Risparmio di tempo: non è necessario eseguire query aggiuntive per ottenere informazioni aggregate in tempo reale.

- Risparmio di spazio: i dati aggregati in tempo reale occupano meno spazio rispetto ai dati non aggregati.

- Maggiore efficienza: i continuous aggregate sono più efficienti dei query batch per l'aggregazione dei dati in tempo reale.

\subsection{Risultati ottenuti}\label{ssec:cont-aggr-risultati}
\ldots