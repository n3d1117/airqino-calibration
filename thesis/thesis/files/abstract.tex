\chapter{Abstract}\label{ch:abstract}
% TODO
Air pollution is currently one of the main issues affecting urbanized areas worldwide. Local administrations monitor these harmful gases by means of reference monitoring stations provided by regional/national environmental protection agencies. These stations, however, have limitations due to coarse spatial coverage of the whole municipality, low time-frequency, and high costs. In this framework, the National Research Council of Italy (CNR-IBE) and the Tuscany Region Environmental Protection Agency (ARPAT) agreed to an initiative to create a low-cost network aimed at monitoring air quality over an Italian urban area. The rural town of Capannori, located in the Tuscany region (Italy), was chosen as a testing area since it lies within a critical area both affected by a variety of emission sources and weather conditions unfavourable to pollutant dispersion. The air quality analysis was carried out by means of several innovative low-cost stations named AIRQino, equipped with sensors for collecting air pollution (PM2.5, PM10, NO2, O3, CO, CO2) and meteorological parameters (air temperature and relative humidity). Concentrations of PM were mainly considered in this work for providing indicative air quality measurements to supplement fixed measurements collected by the official urban monitoring network. This work, still ongoing, has two main objectives: (i) to show the robustness of AIRQino at measuring PM concentrations; (ii) to investigate the PM concentrations dynamics at higher spatial and time scale distribution compared to the reference station.