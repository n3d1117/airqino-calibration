\chapter{Abstract}\label{ch:abstract}

Air pollution is currently one of the main issues affecting urbanized areas worldwide. There is concern regarding the health issues caused by long-term exposure to airborne particulate matter (PM) and other harmful gases (such as \ce{NO_{2}}, \ce{CO_{2}} and \ce{O_{3}}). Measurements at appropriate spatial and temporal scales are essential for understanding and monitoring air pollution, which is required for the development of real-time strategies for exposure control. 
Conventional approaches to air quality monitoring are based on networks of static and sparse measurement stations, provided by regional or national environmental protection agencies. 
These stations, however, have limitations due to coarse spatial coverage of the whole municipality, low time-frequency, and high costs. 
New low-cost and high-portability sensors, intended to complement the existing solutions, are radically changing the conventional approach by allowing real-time information in a high-density form, with new scalable networks (such as AirQino) providing data at fine spatial and temporal scales.

This thesis, developed in collaboration with Magenta srl and the Institute of BioEconomy (IBE) of CNR, focuses on the AirQino platform and has three main objectives: (i) to improve efficiency and scalability of the system; (ii) to investigate and compare different techniques aimed at improving accuracy of the sensors' calibration process; (iii) to develop a web interface to make batch calibration easier.