\chapter{Interfaccia di calibrazione}\label{ch:interfaccia}

% Motivazioni
\section{Motivazioni}\label{sec:motivazioni}
\ldots

% Tecnologie
\section{Tecnologie}\label{sec:tecnologie}
\ldots

\subsection{Backend}\label{ssec:interfaccia-backend}
\ldots

\subsection{Frontend}\label{ssec:interfaccia-frontend}
\ldots

% Funzionamento
\section{Funzionamento}\label{sec:funzionamento}
\ldots

% Autenticazione
\section{Autenticazione}\label{sec:autenticazione}
Keycloak è un'identità federata open source, sviluppata da Red Hat. Può essere utilizzata per gestire l'autenticazione di utenti e servizi in ambienti cloud e on-premise. I principali vantaggi di Keycloak sono la scalabilità, l'affidabilità e la flessibilità.

Keycloak include un server e un agente. L'agente è installato sulle applicazioni che richiedono l'autenticazione, mentre il server gestisce tutte le richieste di autenticazione. Quando un utente tenta di accedere a una applicazione protetta da Keycloak, l'agente verifica se l'utente è autenticato e, in caso affermativo, fornisce le credenziali appropriate all'applicazione.

% CI e deploy automatico
\section{CI e deploy automatico}\label{sec:ci}
Continuous integration è una metodologia di sviluppo software che prevede il continuo e costante integrazione dei cambiamenti effettuati dai developer all'interno di un codice sorgente.

La continuous integration ha lo scopo di evitare problemi di sincronizzazione tra gli sviluppatori, riducendo il numero di bug rilevati in fase di testing e aumentando la qualità del codice prodotto.

I principali vantaggi della continuous integration sono:

- riduzione del numero di bug rilevati in fase di testing;

- aumento della qualità del codice prodotto;

- maggiore sincronizzazione tra gli sviluppatori;

- minor rischio di collisioni tra i cambiamenti effettuati dagli sviluppatori.

Jenkins è uno strumento open source di continuous integration. Jenkins permette di automatizzare il processo di integrazione dei cambiamenti effettuati dai developer all'interno di un codice sorgente, eseguendo una serie di controlli per verificarne la correttezza.

Jenkins può essere utilizzato per gestire una varietà di progetti, tra cui sviluppo software, testing, build, deployment e automazione dei processi.