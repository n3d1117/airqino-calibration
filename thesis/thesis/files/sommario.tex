\chapter{Sommario}\label{ch:sommario}

L'inquinamento atmosferico è uno dei principali problemi che interessano le aree urbanizzate. Tra le conseguenze ci sono i problemi di salute causati dall'esposizione a lungo termine al particolato atmosferico (PM) e ad altri gas nocivi. Continue misurazioni sono essenziali per comprendere e monitorare l'inquinamento atmosferico, necessario per lo sviluppo di strategie in tempo reale per il controllo dell'esposizione.
Gli approcci convenzionali al monitoraggio della qualità dell'aria si basano su reti di stazioni di misurazione fisse, fornite da agenzie di protezione ambientale regionali o nazionali.
Queste stazioni, tuttavia, presentano limitazioni dovute alla copertura spaziale limitata, alla bassa frequenza temporale e ai costi elevati.
Nuovi sensori innovativi a basso costo e ad alta portabilità, integrabili con le soluzioni esistenti, stanno cambiando radicalmente l'approccio convenzionale, consentendo l'invio di informazioni in tempo reale ad alta densità, con nuove reti scalabili (come AirQino) che forniscono dati ad alta risoluzione temporale e spaziale.

Questa tesi, sviluppata in collaborazione con Magenta srl e il Consiglio Nazionale delle Ricerche (CNR-IBE), è incentrata sulla piattaforma AirQino e presenta tre obiettivi principali: (i) migliorare l'efficienza e la scalabilità del sistema; (ii) studiare e confrontare diverse tecniche volte a migliorare l'accuratezza del processo di calibrazione dei sensori; (iii) sviluppare un'interfaccia web per facilitare la calibrazione massiva di centraline.